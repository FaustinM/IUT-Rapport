\documentclass[a4paper, 13pt]{article}
\usepackage[a4paper,top=50pt,bottom=50pt,left=30pt,right=30pt]{geometry}

\usepackage[french]{babel}

\usepackage{mathptmx}
\usepackage{enumitem}
\usepackage{graphicx}

\newcommand{\q}[1]{``#1''}
\begin{document}
\title{Compte rendu de la séance 1}
\author{Ecaterina GUPCA, Alice MERAUD, Jean COSTREL DE CORAINVILLE, Romain REN et Faustin MILLET}
\maketitle

\section{Distinguer les caractéristiques de l'organisation \q{maison d'édition}}
L’organisation est séparée entre plusieurs rôles, plusieurs métiers :
\begin{itemize}
    \item \textbf{L’éditeur} : Acteur central du monde du livre, il accompagne les auteurs, de la rédaction du manuscrit à l’impression jusqu’à la distribution.
    \item \textbf{L’imprimeur} : Il s’occupe de la liaison entre les besoins/envies des clients et ce qui est possible avec les machines. Il gère le processus d’impression (l’offset et le numérique).
    \begin{itemize}
        \item L’offset est le procédé le plus répandu pour l’impression physique et permet de couvrir une gamme de tirages assez larges.
        \item L’impression numérique est le procédé sans imprimante et utilisant les données informatiques, il peut y avoir une impression à la demande qui permet d’imprimer rapidement et à coûts raisonnables ou l’impression à données variables qui permet de personnaliser entièrement et en couleur. Très utilisé dans le marketing.
    \end{itemize}
    \item \textbf{Le distributeur} : Tourné vers la logistique et peut se résumer en quatre verbes : stocker, expédier, retourner et facturer. Il fait en sorte que le livre puisse être acheté partout. Il gère le stock de livres imprimés, il assure le conditionnement et l’envoie par voie postale lors de l’expédition. Il gère aussi le retour des livres (si la mesure est mise en place chez le distributeur), il se charge aussi de la facturation puisqu’il est en contact direct avec les centrales d’achat et les librairies. Il comptabilise les exemplaires vendus et transmet un rapport de vente aux éditeurs.
    \item \textbf{Le diffuseur} : Il s'occupe de toute la stratégie marketing et commercial, il fait en sorte que le livre puisse être vu partout quel que soit le point de vente.
    \begin{enumerate}
        \item Il définit d'abord un objectif de nombre d'exemplaires de livres à insérer sur le marché.
        \item Il démarche ensuite les points de vente pour leur faire connaître le catalogue de votre éditeur et les convaincre de commander des exemplaires.
        \item Il fournit aux points de ventes des supports pour en faire la promotion
        \item Il prend les commandes réceptionnées par les points de vente et les fournit aux éditeurs
        \item Il assure le stock des livres pour éviter les ruptures de stock
        \item Il sensibilise les points de vente sur les nouveautés littéraires de l’éditeur.
    \end{enumerate}
\end{itemize}

\section{Le micro environnement d'une maison d'édition}
\subsection{Les clients}
La clientèle est la cible première de l'entreprise. Le marché peut être complexe et pour atteindre le consommateur final, l'entreprise peut commercialiser ses produits et services auprès d'une clientèle intermédiaire. La clientèle génère une demande explicite ou implicite. Tout l'enjeu réside dans la compréhension des besoins, attentes et motivations afin de proposer une offre pertinente.
\subsection{Les fournisseurs}
Ils exercent une forte influence sur la qualité et la compétitivité de l'offre de l'entreprise. Les prix accordés, la qualité des produits et services livrés, le respect des délais et le support apporté impactent fortement l'entreprise. Il s'agit ici des fournisseurs impliqués en amont de la chaîne de valeur.
\subsection{La concurrence}
Des sociétés concurrentes entrent en rivalité. Aussi, pour développer ou conserver votre leadership, il est indispensable de connaître ses compétiteurs, leurs offres, leurs forces et faiblesses. Ici ce sont les maisons d'éditions
\subsection{Les intermédiaires commerciaux}
Suivant la composition de la filière, un ou plusieurs intermédiaires peuvent intervenir dans l'échange commercial. Agents commerciaux, distributeurs, revendeurs... jouent un rôle central dans la commercialisation de l'offre. Les distributeurs, grossistes et diffuseurs.
\subsection{Les autres partenaires}
Partenaires financiers, conseils, etc, plusieurs acteurs gravitent autour de l'entreprise pour lui fournir des ressources complémentaires : financières, compétences...

\section{Le Macro Environnement d’une maison d’édition}
\subsection{Les facteurs économiques}
\begin{itemize}
    \item Démocratisation des livres numérique
    \item La concurrence entre les maisons des éditions, les auteurs qui sont « volé » par d’autres Éditions.
    \item Des Éditions qui cessent leur activité par exemple « Le Fallois »
    \item Le marché du livre, la première industrie culturelle en France, en pleine croissance. Le secteur de l’édition a ainsi produit, en 2018, 67 942 nouveautés et nouvelles réimpressions (soit 190 nouveautés par jour !), ce qui porte le nombre de références disponibles à 783 000 en version imprimée (+1 \% par rapport à 2017) et à 281 000 en version numérique (+13 \%)
    \item Le prix de vente des livres continue, quant à lui, à progresser nettement moins vite (+0,5 \% en 2017) que l’indice moyen des prix des biens de consommation (+1,8 \%)
    \item Les dimensions, les liens/contacts de la maison d’édition, qui influence sur la publicité du livre.
\end{itemize}
 
\subsection{L'environnement démographique}
\begin{itemize}
    \item La France est un pays des lecteurs.
    \item Une progression de la lecture de 25\% chez l’homme et 37\% chez les gens de 25-34 ans en 2017.
    \item 91\% des Français sont des lecteurs des livres.
\end{itemize}

\subsection{Les facteurs socio-culturels}
\begin{itemize}
    \item Manque de temps a consacré aux loisirs.
    \item 63\% des Français affirme que s’il existait un jour de loisirs supplémentaires, ils le consacrèrent à la lecture !
    \item Le gout de lecture dépend du contexte familial. Si les parents lisent, les enfants devient souvent des grands lecteurs.
\end{itemize}

\subsection{Les facteurs technologiques}
\begin{itemize}
    \item Les réseaux sociaux permettent une meilleure diffusion des livres.
    \item Le marché virtuel facilite l’accès aux livres.
    \item Le scrolling s’intensifie, ce que diminue les heures de lecture.
    \item Le numérique permet de faciliter la lecture (les e-books).
    \item 32\% des femmes de 15-44 ans préfère le numérique.
\end{itemize}

\subsection{Les facteurs politiques, fiscaux et juridiques}
\begin{itemize}
    \item La nouvelle ressource décidée par le gouvernement, le Pass Culture, motive les adolescents à lire.
    \item La loi de 10 août 1981, la loi dite « Lang » protège les libraires d’un possible dumping par les grandes surfaces en imposant un prix unique du livre.
\end{itemize}
\subsection{Les facteurs environnementaux et écologiques}
\begin{itemize}
    \item L’économie du papier (concurrence livre et livre numérique)
    \item Lien éditeur auteur, primordial pour un bon travail 
    \item L’esthétique de couverture et quatrième d’un livre.
\end{itemize}

\section{Fonctionnalités du site \textit{Les éditions Du Détour}}

La page d’accueil du site de la maison d’édition la maison du détour, montre une liste de livre en dessous du bandeau de navigation, on trouve un carrousel avec ce qui semble être ma sélection du moi ou du moment. En dessous on trouve une liste de livre avec la première de couverture, l’auteur et un tableau résumant les données structurelles du livre (ISBN, ou prix par exemple). On peut cliquer sur la couverture ou le titre, pour voir plus de détails sur le document avec un aperçu des premières pages du document en PDF. En cliquant sur une image d’un article, on peut zoomer et la regarder en détail.

Dans le bandeau de navigations on trouve plusieurs catégories qui sont les suivantes, « Apparaître » « Actualités » « Les livres » (la page d’accueil), « Les auteurs » et « La maison ».

Le site met en avant les réseaux sociaux de l’entreprise pour fidéliser leur audience sur d’autres supports. L’éditeur encourage également le partage du site avec des boutons sur le côté et en bas de page, sur les réseaux sociaux et par mail.

Dans la catégorie « À paraître » on nous présente une liste de livre qui vont sortir dans les prochains jours.

Le site ne propose pas un espace de vente en ligne, il ne comporte pas de conditions générales de ventes, ou de fonctionnalités associées.

En bas de page on trouve une barre de recherche permettant à l’utilisateur du site de ne faciliter sa recherche que ce soit dans la liste des livres ou des actualités, pour certains résultats, on a la possibilité de laisser un commentaire. 

En dessous on trouve aussi la possibilité de s’abonner à la newsletter par mail de l’édition du Détour.

Cependant, il est important de noter qu’il n’y a aucune fonctionnalité permettant d’acheter ou de commander directement un exemplaire sur le site.

\end{document}